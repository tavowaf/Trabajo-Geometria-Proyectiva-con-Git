\documentclass[a4,10pt]{article}

\usepackage[utf8]{inputenc}
\usepackage[english,spanish]{babel}
\usepackage{verbatim}
\usepackage{graphicx}

\usepackage{hyperref}

\begin{document}

\title{Aplicaciones de Geometría Proyectiva}
\author{Gustavo Quesada Menchón}
\date{}
\maketitle
\newpage

\tableofcontents

\section{Introducción}

Este documento va ha tratar sobre geometría proyectiva y sus aplicaciones, de cómo con sus operaciones básicas nos facilita la tarea en aplicaciones como el tratamiento de imágenes, reconocimiento facial, visión estereoscópica, etc. Para abordar el tema, se va dividir en varias partes. La parte que nos ocupa, una pequeña introducción con un poco de historia, definiciones y conceptos básicos sobre geometría proyectiva. Una segunda parte donde se explican un poco las principales operaciones que nos puede ser útiles. Una tercera parte donde se profundiza algo más en técnicas aplicadas en el tratamiento de imágenes y visión estereoscópica y una parte final con un pequeño resumen y conclusiones sobre el tema.


%%%%%%%%%%%%%%%%%%%%%%%%%%%%%%%%%%%%%%%%%%%%%%%%%%%%%%%%%%%%%%%%%%%%%%
\section {Definición y conceptos básicos}



%%%%%%%%%%%%%%%%%%%%%%%%%%%%%%%%%%%%%%%%%%%%%%%%%%%%%%%%%%%%%%%%%%%%%%

\section{Aplicaciones en la realidad}



%%%%%%%%%%%%%%%%%%%%%%%%%%%%%%%%%%%%%%%%%%%%%%%%%%%%%%%%%%%%%%%%%%%%%

\section{Conclusión}


\bibliographystyle{plain}
\bibliography{refs}

\end{document}
